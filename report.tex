\documentclass[12pt, a4paper]{article}

% Language

\usepackage[english]{babel}
\usepackage[english]{isodate}
\usepackage[utf8]{inputenc}

% Symbols

\usepackage{amsmath}
\usepackage{amssymb}

% Colors 

\usepackage{xcolor}

% Figures

% \usepackage{float}
% \usepackage{tikz}
% \usepackage{graphicx}
% \usepackage{framed}
% \usepackage{bussproofs}

% Layout

\usepackage[cm]{fullpage}
\setlength{\textheight}{640pt}
% \usepackage{layout}
\usepackage{fancyhdr}

% Misc

\usepackage{listings}
\lstset{
    language=SML
}
\usepackage{alltt}
\usepackage{hyperref}
\hypersetup{
  colorlinks,
  linkcolor={red!50!black},
  citecolor={blue!50!black},
  urlcolor={blue!80!black}
}

% Mathematics

% Formatting 

\setlength{\parindent}{0pt}
\setlength{\parskip}{1em}

% TOC

\setcounter{tocdepth}{3}
\renewcommand{\thesection}{}
\renewcommand{\thesubsection}{}
\renewcommand{\thesubsubsection}{}

% Fancy

\fancypagestyle{plain}{%
\fancyhf{}
\setlength{\headheight}{2.5em}
\setlength{\headsep}{1em}
\renewcommand{\headrulewidth}{0pt}
\fancyhead[R]{%
\small Ass. 3, Week 38; Logic in CS 2014 \\
Karl D. Asmussen, \texttt{twk181}, 1992--11--27}
\fancyfoot[R]{\isodash{--}\isodate\today}
\fancyfoot[L]{Page \thepage\ of \pageref{Fin}}
}
\pagestyle{plain}

% Document proper

\begin{document}
\thispagestyle{empty}

\begin{center}
\textbf{\Huge Compilers}\\
\huge --- Group Assignment ---
\end{center}
\vspace{2em}

\begin{center}
\begin{minipage}{0.7\textwidth}
\LARGE Name: Karl D. Asmussen\\[1em]
Birthday: 1992--11--27\\[1em]
Student identifier: \texttt{twk181} \\[1em]
Faculty: DIKU
\end{minipage}
\end{center}

\pagebreak

\setcounter{page}{1}
\tableofcontents

\pagebreak
%%%%%%%%%%%%%%%%%%%%%%%%%%%%%%%%%%%%%%%%%%%%%%%%%%%%%%%%%%%%%%%%%%%%%%%%%%%%%%%%
%%%%%%%%%%%%%%%%%%%%%%%%%%%%%%%%%%%%%%%%%%%%%%%%%%%%%%%%%%%%%%%%%%%%%%%%%%%%%%%%
%%%%%%%%%%%%%%%%%%%%%%%%%%%%%%%%%%%%%%%%%%%%%%%%%%%%%%%%%%%%%%%%%%%%%%%%%%%%%%%%

\section{Group Dynamics}

My group has unfortunately fallen apart quite hard. First Adrian left due to stress,
then Bertram and myself drifted apart in the fogs of depressive apathy.

Fun.

As a result, I am handing in this assignment myself. I hope that is acceptable.

\section{Optionals and Tests}

I have implemented a solution to the shadowing problem, but not array comprehensions.

My build passes all the included tests, except for the array cromprehension one, and
I have made de-bugging inquries myself (although not implemented them as genuine tests,
because I couldn't make the test-writing work well with my development cycle, primarily
owing to SML's type system ridding me of almost all bugs, and because writing post-hoc
tests is disingenuine.)

\section{No Shortcuts}

For the large part, I have adopted a policy of not using shortcuts of reducing one
structure of FASTO to another in the compilation process. For instance, rewriting
the short-circuiting logic operators into IF-statements. This is largely to challenge
myself.

For Task 1, the implementation of arithmetic operators is fairly simple. In the
interpreter, I default to the builtin operations in SML, but notice in the constant
folding, that I have implemented the common algebraic rules of arithmetic, even
for addition and subtraction.

In the code generation, I found that boolean negation can be efficiently implemented with the
\texttt{slti} instruction, checking if the boolean value is less than 1.

Further, I have hand-implemented the short-circuiting nature of boolean conjunction and disjunction,
althoug I admit I haven't made inquries as to their code-generation friendlyness w.r.t. register
allocation.

A bug that came around to bite me in the arse was code-copying errors. First wrestle I had with
it was when division was translated into multiplication in the Type Checker.

For Task 2, I implemented \texttt{scan} as a straight-forward re-write of \texttt{reduce},
and then modified that to become \texttt{filter}, because \texttt{map}'s code is abhorrently
ugly to poke around.

For filter, I had some trouble with getting correct results, and the following list of
caveats nicely summarizes it:
\begin{itemize}
    \item \textit{Don't muck with the size register, as it is used in determining the
        number of iterations.}

        I had a significant amount of trouble at first, because I thought I could decrement
        the size counter when the predicate return false, thereby retaining a count of
        the number of elements in the new array.

    \item \textit{Remember to increment the iteration-count register before branching
        on the predicate's return value.}

        This led to an interesting series of bugs, where the output array was the
        same size as the input array.
    \item \textit{Adjusting the length of the resultant array can be computed from
        the destination pointer.}

        This allowed me to save a register, and led to another interesting series
        of bugs where the array were four elements longer, and four times larger
        than it should have been.
\end{itemize}

But before any of that, I had the problem that I had copied filter's code off map's
in the type checker, so I spent a good while pondering what was going on, before
figuring out exactly why I got a quite mangled array back.

For Task 3, I decided to write the type-cecking for lambda funargs by hand instead
of deferring to constructing a function declaration. This went well. Compiling the
lambda itself was an interesting mental challenge, but resulted in very little code.

Lastly in Task 4, I noticed while playing around with the \texttt{-p} option on
the fasto compiler, that extraneous \texttt{let} bindings --- those that could
be reduced to a constant or copy-propagation --- were left in, taking up resources
in the code generation, and made it so they were removed while preserving semantic
equivalence.

Implementing shadowing-proofing merely consisted of removing any copied variable's
propagation from the table, thereby allowing for instance, the problematic program
in the report, to compile successfully.

%%%%%%%%%%%%%%%%%%%%%%%%%%%%%%%%%%%%%%%%%%%%%%%%%%%%%%%%%%%%%%%%%%%%%%%%%%%%%%%%
%%%%%%%%%%%%%%%%%%%%%%%%%%%%%%%%%%%%%%%%%%%%%%%%%%%%%%%%%%%%%%%%%%%%%%%%%%%%%%%%
\appendix %%%%%%%%%%%%%%%%%%%%%%%%%%%%%%%%%%%%%%%%%%%%%%%%%%%%%%%%%%%%%%%%%%%%%%
\newpage %%%%%%%%%%%%%%%%%%%%%%%%%%%%%%%%%%%%%%%%%%%%%%%%%%%%%%%%%%%%%%%%%%%%%%%
\phantomsection %%%%%%%%%%%%%%%%%%%%%%%%%%%%%%%%%%%%%%%%%%%%%%%%%%%%%%%%%%%%%%%%
\addcontentsline{toc}{section}{Appendix} %%%%%%%%%%%%%%%%%%%%%%%%%%%%%%%%%%%%%%%
\fancypagestyle{plain}{% %%%%%%%%%%%%%%%%%%%%%%%%%%%%%%%%%%%%%%%%%%%%%%%%%%%%%%%
\fancyhf{} %%%%%%%%%%%%%%%%%%%%%%%%%%%%%%%%%%%%%%%%%%%%%%%%%%%%%%%%%%%%%%%%%%%%%
\setlength{\headheight}{2.5em} %%%%%%%%%%%%%%%%%%%%%%%%%%%%%%%%%%%%%%%%%%%%%%%%%
\setlength{\headsep}{1em} %%%%%%%%%%%%%%%%%%%%%%%%%%%%%%%%%%%%%%%%%%%%%%%%%%%%%%
\renewcommand{\headrulewidth}{0pt} %%%%%%%%%%%%%%%%%%%%%%%%%%%%%%%%%%%%%%%%%%%%%
\fancyhead[R]{% %%%%%%%%%%%%%%%%%%%%%%%%%%%%%%%%%%%%%%%%%%%%%%%%%%%%%%%%%%%%%%%%
\small Ass. 3, Week 38; Logic in CS 2014 \\ %%%%%%%%%%%%%%%%%%%%%%%%%%%%%%%%%%%%
Karl D. Asmussen, \texttt{twk181}, 1992--11--27} %%%%%%%%%%%%%%%%%%%%%%%%%%%%%%%
\fancyfoot[R]{\isodash{--}\isodate\today} %%%%%%%%%%%%%%%%%%%%%%%%%%%%%%%%%%%%%%
\fancyfoot[L]{Page \thepage\ of \pageref{Fin}} %%%%%%%%%%%%%%%%%%%%%%%%%%%%%%%%%
\fancyhead[L]{\scriptsize Appendix} %%%%%%%%%%%%%%%%%%%%%%%%%%%%%%%%%%%%%%%%%%%%
} %%%%%%%%%%%%%%%%%%%%%%%%%%%%%%%%%%%%%%%%%%%%%%%%%%%%%%%%%%%%%%%%%%%%%%%%%%%%%%
\pagestyle{plain} %%%%%%%%%%%%%%%%%%%%%%%%%%%%%%%%%%%%%%%%%%%%%%%%%%%%%%%%%%%%%%
%%%%%%%%%%%%%%%%%%%%%%%%%%%%%%%%%%%%%%%%%%%%%%%%%%%%%%%%%%%%%%%%%%%%%%%%%%%%%%%%
%%%%%%%%%%%%%%%%%%%%%%%%%%%%%%%%%%%%%%%%%%%%%%%%%%%%%%%%%%%%%%%%%%%%%%%%%%%%%%%%

\section{Code Listings of Interest}




% \phantomsection
% \addcontentsline{toc}{section}{List of Figures}
% \listoffigures

% \phantomsection
% \addcontentsline{toc}{section}{List of Tables}
% \listoftables

% \phantomsection
% \addcontentsline{toc}{section}{\bibname}
% \begin{thebibliography}{9}
% \end{thebibliography}

\vfill
\begin{center}
  \textbf{\textit{Fin.}}
  \label{Fin}
\end{center}

\end{document}
